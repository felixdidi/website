%!TEX TS-program = xelatex
%!TEX encoding = UTF-8 Unicode
% Awesome CV LaTeX Template for CV/Resume
%
% This template has been downloaded from:
% https://github.com/posquit0/Awesome-CV
%
% Author:
% Claud D. Park <posquit0.bj@gmail.com>
% http://www.posquit0.com
%
%
% Adapted to be an Rmarkdown template by Mitchell O'Hara-Wild
% 23 November 2018
%
% Template license:
% CC BY-SA 4.0 (https://creativecommons.org/licenses/by-sa/4.0/)
%
%-------------------------------------------------------------------------------
% CONFIGURATIONS
%-------------------------------------------------------------------------------
% A4 paper size by default, use 'letterpaper' for US letter
\documentclass[11pt,a4paper,]{awesome-cv}

% Configure page margins with geometry
\usepackage{geometry}
\geometry{left=1.4cm, top=.8cm, right=1.4cm, bottom=1.8cm, footskip=.5cm}


% Specify the location of the included fonts
\fontdir[fonts/]

% Color for highlights
% Awesome Colors: awesome-emerald, awesome-skyblue, awesome-red, awesome-pink, awesome-orange
%                 awesome-nephritis, awesome-concrete, awesome-darknight

\definecolor{awesome}{HTML}{414141}

% Colors for text
% Uncomment if you would like to specify your own color
% \definecolor{darktext}{HTML}{414141}
% \definecolor{text}{HTML}{333333}
% \definecolor{graytext}{HTML}{5D5D5D}
% \definecolor{lighttext}{HTML}{999999}

% Set false if you don't want to highlight section with awesome color
\setbool{acvSectionColorHighlight}{true}

% If you would like to change the social information separator from a pipe (|) to something else
\renewcommand{\acvHeaderSocialSep}{\quad\textbar\quad}

\def\endfirstpage{\newpage}

%-------------------------------------------------------------------------------
%	PERSONAL INFORMATION
%	Comment any of the lines below if they are not required
%-------------------------------------------------------------------------------
% Available options: circle|rectangle,edge/noedge,left/right

\name{Felix}{Dietrich}

\position{Research and Teaching Associate}
\address{Department of Communication, Johannes Gutenberg University
Mainz, Germany}

\email{\href{mailto:mail@felix-dietrich.de}{\nolinkurl{mail@felix-dietrich.de}}}
\homepage{felix-dietrich.de}
\orcid{0000-0002-0696-3380}
\twitter{Felix\_Dietrich}

% \gitlab{gitlab-id}
% \stackoverflow{SO-id}{SO-name}
% \skype{skype-id}
% \reddit{reddit-id}


\usepackage{booktabs}

\providecommand{\tightlist}{%
	\setlength{\itemsep}{0pt}\setlength{\parskip}{0pt}}

%------------------------------------------------------------------------------



% Pandoc CSL macros
% definitions for citeproc citations
\NewDocumentCommand\citeproctext{}{}
\NewDocumentCommand\citeproc{mm}{%
  \begingroup\def\citeproctext{#2}\cite{#1}\endgroup}
\makeatletter
 % allow citations to break across lines
 \let\@cite@ofmt\@firstofone
 % avoid brackets around text for \cite:
 \def\@biblabel#1{}
 \def\@cite#1#2{{#1\if@tempswa , #2\fi}}
\makeatother
\newlength{\cslhangindent}
\setlength{\cslhangindent}{1.5em}
\newlength{\csllabelwidth}
\setlength{\csllabelwidth}{3em}
\newenvironment{CSLReferences}[2] % #1 hanging-indent, #2 entry-spacing
 {\begin{list}{}{%
  \setlength{\itemindent}{0pt}
  \setlength{\leftmargin}{0pt}
  \setlength{\parsep}{0pt}
  % turn on hanging indent if param 1 is 1
  \ifodd #1
   \setlength{\leftmargin}{\cslhangindent}
   \setlength{\itemindent}{-1\cslhangindent}
  \fi
  % set entry spacing
  \setlength{\itemsep}{#2\baselineskip}}}
 {\end{list}}
\usepackage{calc}
\newcommand{\CSLBlock}[1]{\hfill\break\parbox[t]{\linewidth}{\strut\ignorespaces#1\strut}}
\newcommand{\CSLLeftMargin}[1]{\parbox[t]{\csllabelwidth}{\strut#1\strut}}
\newcommand{\CSLRightInline}[1]{\parbox[t]{\linewidth - \csllabelwidth}{\strut#1\strut}}
\newcommand{\CSLIndent}[1]{\hspace{\cslhangindent}#1}

\begin{document}

% Print the header with above personal informations
% Give optional argument to change alignment(C: center, L: left, R: right)
\makecvheader

% Print the footer with 3 arguments(<left>, <center>, <right>)
% Leave any of these blank if they are not needed
% 2019-02-14 Chris Umphlett - add flexibility to the document name in footer, rather than have it be static Curriculum Vitae
\makecvfooter
  {March 2024}
    {Felix Dietrich~~~·~~~Curriculum Vitae}
  {\thepage}


%-------------------------------------------------------------------------------
%	CV/RESUME CONTENT
%	Each section is imported separately, open each file in turn to modify content
%------------------------------------------------------------------------------



\section{EDUCATION}\label{education}

\begin{cventries}
    \cventry{PhD Student in Communication Science}{Johannes Gutenberg University Mainz}{since 10/21}{}{\begin{cvitems}
\item \textit{Advisor}: Prof. Dr. Leonard Reinecke
\end{cvitems}}
    \cventry{Master of Arts (\textit{with distinction}), Media and Communication Studies — Digital Communication}{University of Mannheim}{09/19-07/21}{}{\begin{cvitems}
\item \textit{Thesis}: Fake news or fake brain? The role of epistemic emotions in the processing of cross-cutting news exposure. \\ \textit{German abstract of the thesis published in \href{http://transfer.dgpuk.de/abstracts/fake-news-oder-fake-brain/}{transfer 25(4)}} \\ \textit{Advisor}: Prof. Dr. Peter Vorderer
\end{cvitems}}
    \cventry{Bachelor of Arts, Media and Communication Studies}{University of Mannheim}{09/16-08/19}{}{\begin{cvitems}
\item \textit{Thesis}: Do you know what the algorithm is doing? The influence of customization affordances on autonomy and the intention to self-disclose in social networks. \\ \textit{Advisor}: Dr. Frank M. Schneider
\end{cvitems}}
\end{cventries}

\section{ACADEMIC POSITIONS}\label{academic-positions}

\begin{cventries}
    \cventry{Research and Teaching Associate at the working group Media Effects and Media Psychology, Department of Communication}{Johannes Gutenberg University Mainz}{since 10/21}{}{\begin{cvitems}
\item \textit{Advisor}: Prof. Dr. Leonard Reinecke
\end{cvitems}}
    \cventry{Teaching Associate at the Institute for Media and Communication Studies}{University of Mannheim}{since 02/22}{}{\begin{cvitems}
\item \textit{Teaching courses in Computational Communication Science for the Mannheim Master in Data Science}
\end{cvitems}}
    \cventry{Research Associate at the Institute for Media and Communication Studies}{University of Mannheim}{08/21-12/21}{}{\begin{cvitems}
\item \textit{Supporting a grant application to the European Research Council}
\end{cvitems}}
\end{cventries}

\section{AWARDS}\label{awards}

\begin{cventries}
    \cventry{together with Rebekka Kreling, Alicia Gilbert, and Leonard Reinecke}{Top Paper Award}{09/23}{}{\begin{cvitems}
\item \textit{at the 13th Conference of the Media Psychology Division of the German Psychological Society (DGPs)}
\end{cvitems}}
\end{cventries}

\section{JOURNAL ARTICLES}\label{journal-articles}

\begingroup
\vspace{0.6em}
\fontsize{9pt}{1em}\bodyfontlight\addfontfeature{StylisticSet={3, 4, 5},Ligatures={Rare,Common,TeX}}\color{text}
\setlength{\parindent}{-0.5in}
\setlength{\leftskip}{0.5in}

\phantomsection\label{refs-d753fc05d33ace184e583e35b4a5dd39}
\begin{CSLReferences}{1}{0}
\bibitem[\citeproctext]{ref-dietrichSurprisedCuriousConfused2024}
\textbf{Dietrich, F.}, Kugler, T., Hennings, S., Conrad, C., Schneider,
F. M., \& Vorderer, P. (2024). Surprised--curious--confused, empathetic,
and entertained? The role of epistemic emotions and empathy in
eudaimonic entertainment experiences and political news processing.
\emph{Media Psychology}, \emph{27}(2), 302--327.
\url{https://doi.org/10.1080/15213269.2023.2236939}

\bibitem[\citeproctext]{ref-gilbertTooAmusedStop2023a}
Gilbert, A., Reinecke, L., Meier, A., Baumgartner, S., \&
\textbf{Dietrich, F.} (2023, September 22). \emph{Too amused to stop?
Self-control and the disengagement process on Netflix}.
\url{https://doi.org/10.31234/osf.io/2xnku}

\bibitem[\citeproctext]{ref-dietrichWhatConstitutesAutonomy2023a}
\textbf{Dietrich, F.}, Arenz, A., \& Reinecke, L. (2023). What
constitutes autonomy in digital communication? A (computational) scoping
review of digital autonomy. \emph{Manuscript Under Review}.

\bibitem[\citeproctext]{ref-schneiderWhatImportantWhen2020}
Schneider, F. M., Domahidi, E., \& \textbf{Dietrich, F.} (2020). What is
important when we evaluate movies? Insights from computational analysis
of online reviews. \emph{Media and Communication}, \emph{8}(3),
153--163. \url{https://doi.org/10.17645/mac.v8i3.3134}

\end{CSLReferences}

\endgroup

\section{CONFERENCE PRESENTATIONS}\label{conference-presentations}

\begingroup
\vspace{0.6em}
\fontsize{9pt}{1em}\bodyfontlight\addfontfeature{StylisticSet={3, 4, 5},Ligatures={Rare,Common,TeX}}\color{text}
\setlength{\parindent}{-0.5in}
\setlength{\leftskip}{0.5in}

\phantomsection\label{refs-74ae8ba0f22a5f51e2ca50112517cc89}
\begin{CSLReferences}{1}{0}
\bibitem[\citeproctext]{ref-dietrichMusicWasMy2024}
\textbf{Dietrich, F.}, Ernst, A., Rohr, B., Scharkow, M., \& Reinecke,
L. (2024, June 20--24). \emph{Music was my first love: An experience
sampling study of biographic resonance through algorithmically curated
music listening} {[}Conference Presentation{]}. 74th Conference of the
International Communication Association (ICA), Gold Coast, Australia.

\bibitem[\citeproctext]{ref-ernstAgencySerendipityKey2024}
Ernst, A., \& \textbf{Dietrich, F.} (2024, June 20--24). \emph{Agency
and serendipity as key concepts for algorithmically curated digital
media use in everyday life} {[}Conference Presentation{]}. 74th
Conference of the International Communication Association (ICA), Gold
Coast, Australia.

\bibitem[\citeproctext]{ref-dietrichOpenSourceTransformer2024}
\textbf{Dietrich, F.}, Possler, D., Lammers, A., \& Scheper, J. (2024,
March 13--15). \emph{Open Source Transformer Modelle: Ein einfaches
Werkzeug zur automatisierten Inhaltsanalyse für die (deutschsprachige)
Kommunikationswissenschaft {[}Open Source Transformer Models: A simple
tool for automated content analysis for (German-speaking) communication
science{]}} {[}Conference Presentation{]}. Annual Conference of the
German Communication Association (DGPuK), Erfurt, Germany.

\bibitem[\citeproctext]{ref-ernstZeitlicheStrukturenDigitaler2024}
Ernst, A., \textbf{Dietrich, F.}, Schnauber-Stockmann, A., Gilbert, A.,
\& Scharkow, M. (2024, January 24--26). \emph{Zeitliche Strukturen
digitaler Unterhaltungsmediennutzung: Eine explorative Analyse digitaler
Verhaltensdaten {[}Temporal structures of digital entertainment media
use: An exploratory analysis of digital trace data{]}} {[}Conference
Presentation{]}. Annual Conference of the Media Use and Effects Division
of the German Communication Association (DGPuK), Fribourg, Switzerland.

\bibitem[\citeproctext]{ref-dietrichUsingLargeLanguage2023}
\textbf{Dietrich, F.} (2023, December 7). \emph{Using large language
models in media psychology} {[}Conference Presentation{]}. Digital
Methods Colloquium at the Weizenbaum Institute, Berlin, Germany.

\bibitem[\citeproctext]{ref-dietrichAlgorithmicallyCuratedMedia2023}
\textbf{Dietrich, F.} (2023, October 5--7). \emph{Algorithmically
curated media entertainment: Insights and open questions for the field
of positive media psychology} {[}Conference Presentation{]}. Positive
Media Psychology Symposium, Orange, California, USA.

\bibitem[\citeproctext]{ref-dietrichDigitalEmotionContagion2023}
\textbf{Dietrich, F.}, Possler, D., \& Dale, K. R. (2023, September
6--8). \emph{Digital emotion contagion in online environments: An
automated content analysis of comments from self-transcendent YouTube
videos} {[}Conference Presentation{]}. 13th Conference of the Media
Psychology Division of the German Psychological Society (DGPs),
Luxembourg.

\bibitem[\citeproctext]{ref-krelingWhatPeopleWatch2023}
Kreling, R., \textbf{Dietrich, F.}, Gilbert, A., \& Reinecke, L. (2023,
September 6--8). \emph{What do people watch under adversity? Testing
interactions of semantic affinity and coping style using Netflix data
donations} {[}Conference Presentation{]}. 13th Conference of the Media
Psychology Division of the German Psychological Society (DGPs),
Luxembourg.

\bibitem[\citeproctext]{ref-rohrOpportunitiesChallengesRealtime2023}
Rohr, B., Ernst, A., \textbf{Dietrich, F.}, \& Scharkow, M. (2023, July
17--21). \emph{Opportunities and challenges of real-time data linkage
designs: A case study using the Spotify API} {[}Conference
Presentation{]}. 10th Conference of the European Survey Research
Association (ESRA), Milan, Italy.

\bibitem[\citeproctext]{ref-dietrichCanGetNo2023}
\textbf{Dietrich, F.}, Ernst, A., Rohr, B., Reinecke, L., \& Scharkow,
M. (2023, May 25--29). \emph{(I can't get no) satisfaction: Music
listeners' algorithmically curated entertainment experience}
{[}Conference Presentation{]}. 73rd Conference of the International
Communication Association (ICA), Toronto, Canada.

\bibitem[\citeproctext]{ref-dietrichWhyAreWe2023}
\textbf{Dietrich, F.}, Hennings, S., \& Vorderer, P. (2023, May 25--29).
\emph{Why are we attracted to true crime? The role of epistemic emotions
and entertainment experiences} {[}Conference Presentation{]}. 73rd
Conference of the International Communication Association (ICA),
Toronto, Canada.

\bibitem[\citeproctext]{ref-dietrichWhatConstitutesAutonomy2023}
\textbf{Dietrich, F.}, \& Reinecke, L. (2023, May 25--29). \emph{What
constitutes autonomy in digital communication? A computational scoping
review of digital autonomy} {[}Conference Presentation{]}. 73rd
Conference of the International Communication Association (ICA),
Toronto, Canada.

\bibitem[\citeproctext]{ref-ernstDigitalJukeboxRevisited2023}
Ernst, A., \textbf{Dietrich, F.}, Rohr, B., \& Scharkow, M. (2023, May
25--29). \emph{The digital jukebox revisited: Applying mood management
theory to algorithmically curated music streaming environments}
{[}Conference Presentation{]}. 73rd Conference of the International
Communication Association (ICA), Toronto, Canada.

\bibitem[\citeproctext]{ref-gilbertTimeWellspentGuilty2023}
Gilbert, A., Reinecke, L., Meier, A., Baumgartner, S. E., Kühne, R., \&
\textbf{Dietrich, F.} (2023, May 25--29). \emph{Time well-spent or
guilty pleasure? The effects of self-control on content selection and
entertainment experience on Netflix} {[}Conference Presentation{]}. 73rd
Conference of the International Communication Association (ICA),
Toronto, Canada.

\bibitem[\citeproctext]{ref-gilbertTooAmusedStop2023}
Gilbert, A., Reinecke, L., Meier, A., Baumgartner, S. E., Kühne, R., \&
\textbf{Dietrich, F.} (2023, January 19--21). \emph{Too amused to stop?
Selbstkontrolle und Unterhaltungserleben bei der Netflix-Nutzung {[}Too
amused to stop? Self-control and entertainment experiences while using
Netflix{]}} {[}Conference Presentation{]}. Annual Conference of the
Media Use and Effects Division of the German Communication Association
(DGPuK), Augsburg, Germany.

\bibitem[\citeproctext]{ref-dietrichRoleEpistemicEmotions2022}
\textbf{Dietrich, F.}, Kugler, T., Hennings, S., Conrad, C., Schneider,
F. M., \& Vorderer, P. (2022, May 26--30). \emph{The role of epistemic
emotions and empathy in eudaimonic entertainment experiences and
political news processing} {[}Conference Presentation{]}. 72nd
Conference of the International Communication Association (ICA), Paris,
France.

\bibitem[\citeproctext]{ref-dietrichRoleEpistemicEmotions2022a}
\textbf{Dietrich, F.}, \& Vorderer, P. (2022, May 26--30). \emph{The
role of epistemic emotions in the processing of cross-cutting news
exposure} {[}Conference Presentation{]}. 72nd Conference of the
International Communication Association (ICA), Paris, France.

\bibitem[\citeproctext]{ref-dietrichAllJustClickbait2021}
\textbf{Dietrich, F.}, Kugler, T., Hennings, S., Conrad, C., Schneider,
F. M., \& Vorderer, P. (2021, September 8--10). \emph{All just
clickbait? The effect of empathy and epistemic emotions in online news
on the eudaimonic entertainment experience and political information
processing} {[}Conference Presentation{]}. 12th Conference of the Media
Psychology Division of the German Psychological Society (DGPs), Aachen,
Germany.

\bibitem[\citeproctext]{ref-vordererHedonicEudaimonicInnovations2019}
Vorderer, P., \& \textbf{Dietrich, F.} (2019, May 24--28).
\emph{Hedonic, eudaimonic, and beyond: Innovations in entertainment
theory} {[}Symposium{]}. 69th Annual Conference of the International
Communication Association (ICA), Washington, D.C., USA.

\bibitem[\citeproctext]{ref-halfmannWhoDeterminesYour2018}
Halfmann, A., Vorderer, P., \textbf{Dietrich, F.}, \& Lutz, S. (2018,
May 24--28). \emph{Who determines your mobile communication? The effects
of social pressure on self-control, need satisfaction, well-being, and
perceived stress} {[}Conference Presentation{]}. 68th Annual Conference
of the International Communication Association (ICA), Prague, Czech
Republic.

\bibitem[\citeproctext]{ref-schmittPopulistVoicesExtremist2018}
Schmitt, J., Winkler, J., Lutz, S., \textbf{Dietrich, F.}, \& Rieger, D.
(2018, May 24--28). \emph{Populist voices in extremist online videos: A
content analysis of right-wing and Islamic extremist YouTube videos}
{[}Conference Presentation{]}. 68th Annual Conference of the
International Communication Association (ICA), Prague, Czech Republic.

\end{CSLReferences}

\endgroup

\section{OTHER PUBLICATIONS}\label{other-publications}

\begingroup
\vspace{0.6em}
\fontsize{9pt}{1em}\bodyfontlight\addfontfeature{StylisticSet={3, 4, 5},Ligatures={Rare,Common,TeX}}\color{text}
\setlength{\parindent}{-0.5in}
\setlength{\leftskip}{0.5in}

\phantomsection\label{refs-1ad97b12bb8b22e86e46411237317ecb}
\begin{CSLReferences}{1}{0}
\bibitem[\citeproctext]{ref-dietrichPotenzialDigitalerUnterhaltungsangebote2023}
\textbf{Dietrich, F.} (2023). Das Potenzial digitaler
Unterhaltungsangebote in algorithmisch kuratierten Onlineumgebungen für
den öffentlich-rechtlichen Programmauftrag {[}The potentials of digital
entertainment programming in algorithmically curated online environments
for the public service broadcasting mandate{]}. In \emph{Public Value
Studie: Die Bedeutung öffentlich-rechtlicher Unterhaltung in Zeiten des
digitalen Wandels {[}Public Value Study: The importance of entertainment
programming for public service broadcasting in times of digital
transformation{]}} (pp. 104--118). Österreichischer Rundfunk (ORF).
\url{https://zukunft.orf.at/}

\bibitem[\citeproctext]{ref-dietrichSocialMediaAffordances2022}
\textbf{Dietrich, F.}, \& Reinecke, L. (2022). Social media affordances
and well-being: An integration with HCI-research. In N. Ballou, S.
Deterding, A. Tyack, E. D. Mekler, R. A. Calvo, D. Peters, G.
Villalobos-Zúñiga, \& S. Turkay (Eds.), \emph{Self-determination theory
in HCI: Shaping a research agenda {[}Workshop presentation at the CHI
Conference on Human Factors in Computing Systems{]}}.
\url{https://www.positivecomputing.org/blog/chi-2022-workshop}

\bibitem[\citeproctext]{ref-dietrichFakeNewsOder2021}
\textbf{Dietrich, F.} (2021). Fake News oder Fake Brain? Die Rolle
epistemischer Emotionen bei der Rezeption von politischen Nachrichten,
die der eigenen Meinung widersprechen {[}Fake news oder fake brain? The
role of epistemic emotions in the processing of cross-cutting news
exposure{]}. \emph{transfer}, \emph{25}(4).
\url{http://transfer.dgpuk.de/abstracts/fake-news-oder-fake-brain/}

\end{CSLReferences}

\endgroup

\section{TEACHING EXPERIENCE}\label{teaching-experience}

\begin{cventries}
    \cventry{Johannes Gutenberg University Mainz}{The Algorithm Knows Me (Not): Opportunities and Risks of Algorithmic Curation of Entertainment Media}{04/24 - 09/24}{}{\begin{cvitems}
\item Seminar in the module ``New Media and Online Communication'' for the BA Communication Science program
\end{cvitems}}
    \cventry{University of Mannheim}{Computational Analysis of Communication}{02/24 - 07/24}{}{\begin{cvitems}
\item Seminar in the module ``Data Analytics Methods'' for the MA Data Science program
\end{cvitems}}
    \cventry{Johannes Gutenberg University Mainz}{Core Concepts and Theories of Communication}{10/23 - 03/24}{}{\begin{cvitems}
\item Seminar in the module ``Fundamentals of Communication Science'' for the BA Communication Science program
\end{cvitems}}
    \cventry{University of Mannheim}{Computational Analysis of Communication}{08/23 - 01/24}{}{\begin{cvitems}
\item Seminar in the module ``Data Analytics Methods'' for the MA Data Science program
\end{cvitems}}
    \cventry{Johannes Gutenberg University Mainz}{What Constitutes Autonomy in Digital Communication?}{04/23 - 09/23}{}{\begin{cvitems}
\item Seminar in the module ``New Media and Online Communication'' for the BA Communication Science program
\end{cvitems}}
    \cventry{University of Mannheim}{Computational Analysis of Communication}{02/23 - 07/23}{}{\begin{cvitems}
\item Seminar in the module ``Data Analytics Methods'' for the MA Data Science program
\end{cvitems}}
    \cventry{Johannes Gutenberg University Mainz}{Academic Reading, Comprehension and Writing}{10/22 - 03/23}{}{\begin{cvitems}
\item Seminar in the module ``Fundamentals of Communication Science'' for the BA Communication Science program
\end{cvitems}}
    \cventry{University of Mannheim}{Computational Analysis of Communication}{08/22 - 01/23}{}{\begin{cvitems}
\item Seminar in the module ``Data Analytics Methods'' for the MA Data Science program
\end{cvitems}}
    \cventry{Johannes Gutenberg University Mainz}{Intended and Unintended Side Effects of Media Entertainment}{04/22 - 09/22}{}{\begin{cvitems}
\item Seminar in the module ``Media Effects Research'' for the BA Communication Science program
\end{cvitems}}
    \cventry{Johannes Gutenberg University Mainz}{Self-Regulatory Chances and Risks of Permanent Connectedness}{04/22 - 09/22}{}{\begin{cvitems}
\item Seminar in the module ``New Media and Online Communication'' for the BA Communication Science program
\end{cvitems}}
    \cventry{University of Mannheim}{Automated Media Content Analysis}{02/22 - 07/22}{}{\begin{cvitems}
\item Seminar in the module ``Data Analytics Methods'' for the MA Data Science program
\end{cvitems}}
    \cventry{University of Mannheim}{Introduction to Scientific Working}{08/20 - 01/21}{}{\begin{cvitems}
\item Student-led exercise for the BA Media and Communcation Studies program
\end{cvitems}}
    \cventry{University of Mannheim}{Introduction to Scientific Working}{08/19 - 01/20}{}{\begin{cvitems}
\item Student-led exercise for the BA Media and Communcation Studies program
\end{cvitems}}
    \cventry{University of Mannheim}{Introduction to Scientific Working}{08/18 - 01/19}{}{\begin{cvitems}
\item Student-led exercise for the BA Media and Communcation Studies program
\end{cvitems}}
    \cventry{University of Mannheim}{Introduction to Scientific Working}{08/17 - 01/18}{}{\begin{cvitems}
\item Student-led exercise for the BA Media and Communcation Studies program
\end{cvitems}}
\end{cventries}

\section{EARLY ACADEMIC EXPERIENCE}\label{early-academic-experience}

\begin{cventries}
    \cventry{University of Mannheim}{Research Assistant}{04/17 - 07/21}{}{\begin{cvitems}
\item Institute for Media and Communication Studies (Media Psychology), Prof. Dr. Peter Vorderer
\end{cvitems}}
    \cventry{University of Mannheim}{Teaching Assistant}{02/21 - 07/21}{}{\begin{cvitems}
\item Practical seminar II: Job-related Project Seminar, Dr. Dorothée Hefner
\end{cvitems}}
    \cventry{University of Mannheim}{Teaching Assistant}{02/20 - 07/20}{}{\begin{cvitems}
\item Practical seminar II: Job-related Project Seminar, Prof. Dr. Angela Keppler
\end{cvitems}}
    \cventry{University of Mannheim}{Teaching Assistant}{02/19 - 07/19}{}{\begin{cvitems}
\item Practical seminar II: Job-related Project Seminar, Prof. Dr. Angela Keppler
\end{cvitems}}
    \cventry{University of Mannheim}{Research Assistant}{07/17 - 10/17}{}{\begin{cvitems}
\item CONTRA: Countering Propaganda by Narration Towards Anti-Radical Awareness, funded by the European Commission
\end{cvitems}}
\end{cventries}

\section{PRACTICAL EXPERIENCE}\label{practical-experience}

\begin{cventries}
    \cventry{Bergsträßer Anzeiger}{Freelance Journalist}{2013-2015}{}{\begin{cvitems}
\item Bensheim, Germany
\end{cvitems}}
    \cventry{TasteNext gUG}{Public Relations \& Concept Development}{2014-2018}{}{\begin{cvitems}
\item Mannheim, Germany
\end{cvitems}}
    \cventry{delicom S.L.}{Public Relations \& Concept Development}{2015-2020}{}{\begin{cvitems}
\item Madrid, Spain
\end{cvitems}}
\end{cventries}

\section{LANGUAGES}\label{languages}

\begin{cventries}
    \cventry{Native Speaker}{German}{}{}{}\vspace{-4.0mm}
    \cventry{Fluent}{English}{}{}{}\vspace{-4.0mm}
    \cventry{HSK Level III}{Mandarin Chinese}{}{}{}\vspace{-4.0mm}
    \cventry{Latinum}{Latin}{}{}{}\vspace{-4.0mm}
\end{cventries}

\section{PROGRAMMING LANGUAGES}\label{programming-languages}

\begin{cventries}
    \cventry{advanced}{R}{}{}{}\vspace{-4.0mm}
    \cventry{advanced}{HTML}{}{}{}\vspace{-4.0mm}
    \cventry{advanced}{CSS}{}{}{}\vspace{-4.0mm}
    \cventry{basic}{Python}{}{}{}\vspace{-4.0mm}
    \cventry{basic}{LaTeX}{}{}{}\vspace{-4.0mm}
    \cventry{basic}{Bash}{}{}{}\vspace{-4.0mm}
\end{cventries}



\end{document}
